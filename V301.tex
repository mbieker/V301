\documentclass[11pt,ngerman,a4paper]{article}
%Gummi|061|=)
\usepackage{amsmath}
\usepackage{a4wide}
\usepackage{amsthm}
\usepackage{amsbsy}
\usepackage{amssymb}
\usepackage{inputenc}
\usepackage{rotating} 
\usepackage{graphicx}
\usepackage{paralist}
\usepackage{selinput}
\SelectInputMappings{
adieresis={ä},
germandbls={ß},
}

\title{\textbf{Versuch V301: EMK und Innenwiderstand von Spannungsquellen}}
\author{Martin Bieker \\ \textit{martin.bieker@udo.edu} \and Julian Surmann \\ \textit{julian.surmann@udo.edu}}
\date{}
\usepackage{graphicx}
\begin{document}
\renewcommand\tablename{Tabelle}
\renewcommand\figurename{Abbildung}
\maketitle
\thispagestyle{empty}
\begin{center} Durchgeführt am 28.11.2013\\ TU Dortmund \end{center}


\newpage
\clearpage
\setcounter{page}{1}

\section{Einleitung}
In diesem Versuch sollen die Eigenschaften realer Spannungsquellen betrachtet werden. Von besonderer Bedeutung sind hier die Leerlaufspannung und der Innenwiderstand.
\section{Theorie}
Als Spannungsquelle beschreibt man ein elektrisches Bauteil mit zwei Anschl\"ussen, das zwischen diesen Polen \"uber einen endlichen Zeitraum eine gleichm\"a\ss ige Spannung bereitstellt. Bei einer idealen Spannungsquelle h\"angt diese Spannung nicht von entnommenenem Strom ab.
\begin{figure}[htp]
\centering
\includegraphics[scale=1.00]{abb3.png}
\caption{Ersatzschaltung f\"ur eine reale Spannungsquelle [1]}
\label{Ersatz}
\end{figure}
Eine reale Spannungsquelle kann  durch eine Ersatzschaltung aus einer idealen Spannungsquelle und einem Widerstand $R_i$ dargestellt werden (siehe Abb. \ref{Ersatz}). Hierbei ist $R_i$ der Innenwiderstand der Quelle und  $U_0$ die sogenannte Leerlaufspannung. Dies ist die Spannung, die zwischen den Polen anliegt, wenn aus der Spannungsquelle kein Strom entnommen wird. $U_k$ ist die Klemmenspannung, die zwischen den Polen der belasteten Quelle anliegt. Gem\"a\ss\ der Maschenregel (2. Kirchhoffsches Gesetz)
\begin{equation}
0 = -U_0 + U_k + U_{Ri}
\end{equation}
und dem Ohm'schen Gesetz
\begin{equation}
U_{Ri} = R_i \cdot I
\end{equation}
gilt f\"ur die Klemmenspannung
\begin{equation}
U_k = U_0 - R_i \cdot I.
\label{Klemmenspannung}
\end{equation}
Hieraus folgt, dass die Klemmenspannung einer realen Spannungsquelle bei Belastung kleiner ist als die Leerlaufspannung. Eine ideale Spannungsquelle dagegen hat keinen Innenwiderstand. Deshalb ist die Klemmenspannung in diesem Fall unabh\"angig von der Belastung der Quelle.
Des Weiteren wird aus Gleichung (\ref{Klemmenspannung}) ersichtlich, dass man einer realen Spannungsquelle nur eine begrenzte Leistung $P_{Max}$ entnehmen kann. Aus 
\begin{equation}
P = U \cdot I = R_a \cdot I^2 = \frac{U_0^2 \cdot R_a}{(R_a+R_i)^2}
\label{erw_leistung}
\end{equation}
folgt f\"ur den Lastwiderstand $R_a$
\begin{equation}
\frac{dP}{dt} = 0 \rightarrow R_a = R_i
\label{leistungsanpassung}
\end{equation}
mit einer der maximalen Leistung 
\begin{equation}
P_{Max} = \frac{U_0^2}{4\cdot R_i}.
\end{equation}

\section{Versuchsdurchf\"uhrung}
Im diesen Versuch werden $U_0$ und $R_i$ f\"ur folgende Spannungsquellen bestimmt. 
\begin{itemize}
\item Monozelle
\item RC-Generator (Rechteckspannung)
\item RC-Generator (Sinusspannung)
\end{itemize}
\subsection{Direkte Messung der Leerlaufspannung}
Die Klemmenspannung wird zun\"achst mit einem Voltmeter gemessen. Da dieses einen hohen Eingangswiderstand hat, flie\ss t nur ein geringer Strom und in Formel (\ref{Klemmenspannung}) kann der Term $R_i \cdot I$ vernachl\"assigt werden, sodass gilt: 
\begin{equation}
U_0 \approx U_k.
\end{equation} 

\subsection{Messung an der belasteten Spannungsquelle}
Um den Innenwiderstand $R_i$ und nochmals die Leerlaufspannung $U_0$ zu messen wird an die Spannungsquelle ein variabler Lastwiderstand $R_a$ angeschlossen und die Klemmensapnnung $U_k$ sowie der entnommene Strom $I$ gemessen. Abbildung \ref{Aufbau1} zeigt den verwendeten Aufbau. 
\begin{figure}[htp]
\centering
\includegraphics[scale=1.00]{abb1.png}
\caption{Versuchsaufbau ohne Gegenspannung [1]}
\label{Aufbau1}
\end{figure}
Die Gr\"o\ss e des Lastwiderstandes wird gleichm\"a\ss ig in den folgenden Bereichen variiert:
\begin{table}[h!]
\centering
\begin{tabular}{|c|c|c|}
\hline
Spannungsquelle& $R_{Min} [\Omega]$& $R_{Max} [\Omega]$\\
\hline
Monozelle & 0& 50\\
Rechteckspannung &  20 & 250\\
Sinusspannung&100& 5000\\
\hline
\end{tabular}

\caption{Wertebreiche das Lastwiderstands $R_a$ f\"ur verschiedene Spannungsquellen}
\end{table}\\
Um eine \"uberm\"assige Belastung der Spannungsquellen, vor allem bei niedrigen Lastwiderst\"anden zu verhindern, wird der Stromkreis mit dem Taster nur zur Messung geschlossen.
\subsection{Messung mit Gegenspannung}
Im letzten Versuchsteil wird zus\"atzlich eine Gegenspannung 
\[
U_g = 3.58 V 
\] an die Monozelle angelegt (siehe Abb. \ref{Aufbau2}). 
\begin{figure}[htp]
\centering
\includegraphics[scale=1.00]{abb2.png}
\caption{Versuchsaufbau mit Gegenspannung [1]}
\label{Aufbau2}
\end{figure}Analog zu oben werden der f\"ur 11 verschiedene Lastwiderst\"ande von $0\,\Omega$ bis $100\,\Omega$ die Klemmenspannung und der flie\ss ende Strom gemessen. Da $U_g \gg U_0$ ist, durchläuft der Strom $I$ die Masche in entgegengesetzter Richtung. Gemäß der Maschenregel gilt für $U_k$. 
\begin{equation}
U_k = U_0 + I\cdot R_i.
\end{equation}
\newpage
\section{Auswertung}


\subsection{Bestimmung von $U_0$ und $R_i$}
Zur Bestimmung des Innenwiderstandes und der Leerlaufspannung werden die gemessenen Spannungswerte f\"ur $U_k$ gegen den Strom $I$ aufgetragen. Abbildung \ref{Plot1} zeigt den Verlauf f\"ur die Monozelle, Abbildung \ref{Plot2} f\"ur die Rechteckspannung und Abbildung \ref{Plot3} f\"ur die Sinusspannung. Alle Messwerttabellen des Versuches befinden sich im Anhang.


 

\begin{sidewaysfigure}[p]
\centering
\includegraphics[scale=1.00]{Plot1.png}
\caption{$U_k$ in Abh\"angigkeit von $I$ f\"ur die Monozelle (ohne Gegenspannnung)}
\label{Plot1}
\end{sidewaysfigure}
\begin{sidewaysfigure}[p]
\centering
\includegraphics[scale=1.00]{Plot5.png}
\caption{$U_k$ in Abh\"angigkeit von $I$ f\"ur die Monozelle (mit Gegenspannnung)}
\label{Plot1}
\end{sidewaysfigure}
\begin{sidewaysfigure}[p]
\centering
\includegraphics[scale=1.00]{Plot2.png}
\caption{$U_k$ in Abh\"angigkeit von $I$ f\"ur die Rechteckspannung}
\label{Plot2}
\end{sidewaysfigure}
\begin{sidewaysfigure}[p]
\centering
\includegraphics[scale=1.00]{Plot3.png}
\caption{$U_k$ in Abh\"angigkeit von $I$ f\"ur die Sinusspannung}
\label{Plot3}
\end{sidewaysfigure}

\noindent
Mit Hilfe der Linearen Regression können jeweils der Innenwiderstand und die Leerlaufspannung der Spannungsquellen bestimmt werden. Folgende Werte wurden mit Python ermittelt:

 \begin{table}[h]
 \centering
 \begin{tabular}{|c|c|c|}
  \hline
  Spannungsquelle & $R_I[\Omega]$ & $U[V]$ \\
  \hline
  Monozelle & $11.58\pm0.05$ & $1.690\pm0.003$ \\
  Monozelle mit Gegenspannung  & $10.0 \pm 0.3$ &$1.76\pm0.02$\\
  Rechteckspannung & $61.3\pm0.9$& $0.482\pm0.003$ \\
  Sinusspannung & $730\pm10$ & $1.692\pm0.007$\\
  \hline
 \end{tabular}
 \caption{Ergebnis - Innenwiderstand und Leerlaufspannung}
 \label{Ergebnis}
 \end{table}

\subsection{Direkte Messung von $U_0$}
Die direkte Messung von $U_0$ der Monozelle ergab eine Leerlaufspannung von $1.68\, V$. Der Fehler der Leerlaufspannung ergibt sich aus einer kurzen Herleitung:\newline
Aus Formel (3) ergibt sich
\[ U_0 = U_k + I \cdot R_v. \]
Mit $I=U_k / R_v$ ist
\[ U_0 = U_k + \frac{U_k \cdot R_i}{R_v}.\]
Durch Ausklammern von $U_k$ erhält man
\[ U_0 = U_k \left( 1+\frac{R_i}{R_v}\right) .\]
Nun ergibt sich der Fehler der Leerlaufspannung als
\[ \Delta U = \frac{U_k \cdot R_i}{R_v}. \]
Der Fehler der Leerlaufspannung der Monozelle beträgt jetzt
\[ \Delta U_0 = 1.946 \cdot 10^{-6}.\]
\newline
Falls man das Voltmeter für die Messung der Spannung in der Schaltung hinter das Amperemeter setzt, macht man systematische Fehler. So zeigt eine Betrachtung der Maschenregel
\[ 0 = -U_0 + U_k + R_i \cdot I + R_{Amp} \cdot I ,\]
dass an dem Amperemeter auch ein Teil der Spannung abfällt:
\[ U_k = U_0 - I(R_i + R_{Amp}). \]
\newline
Die mit dem Voltmeter gemessene Leerlaufspannung wird daher kleiner.
\subsection{Leistung der Monozelle}

Zur Betrachtung des Leistungsverlauf wird die am jeweiligen Lastwiderstand 
\begin{equation}
R_a = U_k \cdot I
\end{equation}
 abfallende Leistung
\begin{equation}
P = U_a\cdot  I
\end{equation}
berechnet (siehe Tabelle \ref{leistung}). In Abbildung \ref{Plot4} sind dieser Werte gegeneinander aufgetragen. In diesem Diagramm ist auch die mit Formel (\ref{erw_leistung}), $U_0$ und $R_i$ berechneten Werte f\"ur die zu erwartende Leistung aufgetragen.
\begin{sidewaysfigure}[htp]
\centering
\includegraphics[scale=1.00]{Plot4.png}
\caption{Leistungsverlauf an der Monozelle}
\label{Plot4}
\end{sidewaysfigure}
Es ist zu erkennen, dass die Abweichung der Messwerte von der berechneten Kurve im Rahmen der Messungenauigkeit liegt. Des Weiteren ist erkennbar, dass das Maximum der von $R_a$ aufgenommen Leistung bei 
\[
R_a \approx R_i
\]
liegt. Dies best\"atigt die Berechnungen im Rahmen der Leistungsanpassung (siehe Formel (\ref{leistungsanpassung})).
\newpage
\section{Diskussion}
Die in diesem Versuch verwendeten Messgeräte haben eine Messgenauigkeit von $\pm 3\,\%$. Da aber Leerlaufspannung und Innenwiderstand durch eine Ausgleichsrechnung bestimmt wurden, kann die Messunsicherheit der Multimeter gegenüber den Regressionsfehlern vernachlässigt werden. 
Des Weiteren hat der Versuch gezeigt , dass bei Spannungsmessgeräten mit einem hinreichend großen Eingangswiderstand der Fehler bei einer direkten Messung der Leerlaufspannung vernachlässigt werden kann.
\section{Quellenverzeichnis}
\begin{enumerate}[{[}1{]}]

\item \textit{Leerlaufspannung und Innenwiderstand von Spannungsquellen}, Physikalisches Praktikum\\ TU Dortmund
\end{enumerate}

\section{Anhang}
\begin{itemize}

\item Tabelle 3: Messung an der Monozelle
\item Tabelle 4: Messung an der Monozelle mit Gegenspannung
\item Tabelle 5: Messung mit Rechteckspannung
\item Tabelle 6: Messung mit Sinusspannung
\item Tabelle 3: Leistungswerte der Monozelle
\item Kopien des Messheftes
\end{itemize}
\newpage
\begin{table}[h]
 \centering

 \begin{tabular}{|c|c|}
  \hline
  $U[V]$ & $I[mA]$  \\
  \hline
  1.42 & 23 \\
  1.395 & 25.9 \\
  1.37& 28.5 \\
  1.32 & 31.9 \\
  1.30 & 33.5\\
  1.23 & 39.5\\
  1.16 & 45 \\
  1.04 & 56\\
  0.86 & 72\\
  0.61 & 93\\
  0.082 & 139\\
  \hline
 \end{tabular}
  \caption{Messung an der Monozelle}
 \label{Messung 1}
 \end{table}
 
 
 \begin{table}[h]
 \centering
 \begin{tabular}{|c|c|}
  \hline
  $U[V]$ & $I[mA]$  \\
  \hline
  2.07 & 34.5 \\
  2.09 & 36 \\
  2.15 & 40 \\
  2.19 & 43 \\
  2.26 & 48.5\\
  2.35 & 55.5\\
  2.42 & 63 \\
  2.58 & 77.5\\
  2.718 & 93.5\\
  2.91 & 125\\
  3.73 & 195\\
  \hline
 \end{tabular}
  \caption{Messung an der Monozelle mit Gegenspannung}
 \label{Messung 2}
 \end{table}
 
 
 \begin{table}[h]
 \centering

 \begin{tabular}{|c|c|}
  \hline
  $U[V]$ & $I[mA]$  \\
  \hline
  0.395 & 1.4 \\
  0.39 & 1.46 \\
  0.3825& 1.61 \\
  0.375 & 1.75 \\
  0.3625 & 1.97\\
  0.3475 & 2.25\\
  0.3275 & 2.62 \\
  0.3 & 3.05\\
  0.2525 & 3.6\\
  0.19 & 4.7\\
  0.105 & 6.2\\
  \hline
 \end{tabular}
  \caption{Messung mit Rechteckspannung}
 \label{Messung 3}
 \end{table}
 
 
 \begin{table}[h]
 \centering

 \begin{tabular}{|c|c|}
  \hline
  $U[V]$ & $I[mA]$  \\
  \hline
  1.525 & 0.23 \\
  1.5125 & 0.245 \\
  1.5& 0.2725 \\
  1.48 & 0.296 \\
  1.455 & 0.305\\
  1.42& 0.361\\
  1.37 & 0.44 \\
  1.28 & 0.57\\
  1.13 & 0.81\\
  0.84 & 1.13\\
  0.5 & 1.64\\
  \hline
 \end{tabular}
  \caption{Messung mit Sinusspannung}
 \label{Messung 4}
 \end{table}
 
 \begin{table}
 \centering
 \begin{tabular}{|c|c|}
\hline
$R_a [\Omega]$ & $N [W]$ \\
\hline
61.7 & 0.033\\
53.9 & 0.036\\
48.1 & 0.039\\
41.4 & 0.042\\
38.8 & 0.044\\
31.1 & 0.049\\
25.8 & 0.052\\
18.6 & 0.058\\
11.9 & 0.062\\
6.6 & 0.057\\
0.6 & 0.011\\
\hline
\end{tabular}
\caption{Leistungswerte der Monozelle}
\label{leistung}
\end{table}

\end{document}
