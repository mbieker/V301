\documentclass[11pt]{article}
%Gummi|061|=)
\usepackage{amsmath}
\usepackage{amsthm}
\usepackage{amsbsy}
\usepackage{amssymb}
\usepackage{inputenc}
\usepackage{graphicx}
\usepackage{selinput}
\usepackage{here}
\SelectInputMappings{
adieresis={ä},
germandbls={ß},
}
\title{\textbf{Versuch V204: Wärmeleitung von Metallen}}
\author{Martin Bieker\\
		Julian Surmann\\
		\\
		Durchgef\"{u}hrt am 31.10.2013\\
		Tu Dortmund}
\date{}
\usepackage{graphicx}
\begin{document}
\renewcommand\tablename{Tabelle}
\renewcommand\figurename{Abbildung}
\maketitle
\thispagestyle{empty}
\newpage
\clearpage
\setcounter{page}{1}

\section{Einleitung}
In diesem Versuch sollen die Eigenschaften realer Spannungsquellen betrachtet werden. Von besonderer Bedeutung sind hier die Leerlaufspannung und der Innenwiderstand der Quellen.
\section{Theorie}
Als Spannungsquelle beschreibt man ein 
\section{Versuchsdurchführung}
\section{Auswertung}
\section{Diskussion}
\end{document}
